
\title{Probabilistic Audio Localization in Wireless Acoustic Sensor Networks}

\date{\today}

\author{Alexander Wallar}

\documentclass[12pt]{article}

\usepackage[pdftex]{graphicx}

\usepackage{amsmath}

\usepackage{relsize}

\usepackage{float}

\usepackage{algorithm}

\usepackage[noend]{algorithmic}

\floatstyle{ruled} \newfloat{program}{thp}{lop} \floatname{program}{Structure}

\newcommand{\Normal}[3]{\mathcal{N}(#1, #2, #3)}

\newcommand{\Acronym}[1]{\ensuremath{{\small{\texttt{#1}}}}}
\newcommand{\Name}{\Acronym{Camgaze.js}} \newcommand{\False}{\Constant{false}}
\newcommand{\True}{\Constant{true}}
\newcommand{\Symbol}[1]{\ensuremath{\mathcal{#1}}}
\newcommand{\Function}[1]{\ensuremath{{\small \textsc{#1}}}}
\newcommand{\Constant}[1]{\ensuremath{\small{\texttt{#1}}}}
\newcommand{\Var}[1]{\ensuremath{{\small{\textsl{#1}}}}}
\newcommand{\argmin}[1]{\underset{#1}{\operatorname{arg}\,\operatorname{min}}\;}
\newcommand{\grad}[1]{\underset{#1}{\operatorname{\Function{GradientDecent}}}\;}

\begin{document}

\maketitle

\section{Implementation}

\subsection{Acquisition}

To determine the position of a sound within a wireless acoustic sensor network
(WASN), we must discuss the structure and type of data that is being sent from
the acoustic nodes. We define a \emph{node event}, which is the structure being
sent from the acoustic nodes to central server. Below is an outline of a node
event.


\begin{program}
\caption{Node Event Structure}
\begin{verbatim}
{
    x: <Float: X position>,
    y: <Float: Y position>,
    spl: <Float: Sound pressure level>,
    timestamp: <Float: Unix time in seconds>,
    confidence: <Float: Confidence of recognition>
    name: <String: Name of the detected sound in the database>
}
\end{verbatim}
\end{program}

The node event structure provides all of the acquisition data needed by the
central server in order to perform audio localization. This structure is sent
in JSON to the central server using a \verb|POST| request upon the positive
recognition of a sound in the server's database. The recognition is done on the
wireless acoustic sensor node.

\subsection{Localization}

\subsubsection{Single Source}

In order to conduct localization using the node events, we have created a
probability distribution that given an $x$ and $y$ position along with a list
of node events and reference data regarding the detected sound (such as sound
pressure level and distance at time of recording), we can deduce the
probability of the sound being at position $(x, y)$. The probability function
is fairly complex and is broken down into subfunctions below.

Firstly, at the heart of the probability function is a normal distribution
denoted as $\Normal{X}{\mu}{\sigma}$, where $X$ is the distance from the node
event, $\mu$ is the predicted distance from the node that the sound will be,
and $\sigma$ is how confident the recognition is. To determine the mean for the
normal distribution, we used the predicted distance from node that the sound
will be given the reference sound pressure level and distance at which the
reference data was taken as well as the acquired sound pressure level. With
these three pieces of data, we are able to predict the distance away from the
microphone the sound was. This gives us the mean value for the normal
distribution. The function is shown below.  \begin{align*}
\Function{D}_{s}(r, spl, spl') &= r \cdot 10 ^ {\frac{(spl - spl')}{20}}
\end{align*}

In the equation above, $r$ is the reference distance at time of recording,
$spl$ is the reference sound pressure level at the time of recording, and
$spl'$ is the detected sound pressure level. The result is the corresponding
distance for the input sound pressure level. The input, $X$, to the normal
distribution function is the euclidean distance from the node event to the an
$(x, y)$ position. The formula is shown below.  \begin{align*}
\Function{D}(x, y, n) &= \sqrt{(x - n.x) ^ 2 + (y - n.y) ^ 2} \end{align*}

Lastly, to complete the parameters for the normal distribution, we need to
determine what standard deviation to use for each of the node events. To
determine the standard deviation, we use the timestamps and confidence of
recognition available from the node events. This is described more in
\textbf{Algorithm} \ref{algo:Std}

\begin{algorithm}[ht] \caption{Determining the Standard Deviation}
\label{algo:Std} \begin{algorithmic}[1] \setcounter{ALC@line}{0}

\vspace*{1mm}

\STATE $\Function{GetSD}(\Var{event}, \Var{eventList}) \rightarrow$
\STATE $t_{max} \leftarrow \Function{GetMaxTime}(\Var{eventList})$
\STATE $t_{min} \leftarrow \Function{GetMinTime}(\Var{eventList})$
\STATE $t_{err} \leftarrow 1 - \mathlarger{\frac{t_{max} - \Var{event}.\Var{timestamp}}{t_{max} - t_{min} + \epsilon}}$
\STATE $\Var{err} \leftarrow \frac{c}{t_{err} + \Var{event}.\Var{confidence}}$
\COMMENT {$c$ is a scaling constant}
\RETURN $\Var{err}$

\end{algorithmic}
\end{algorithm}

Finally, we are able to derive the function for sound position probability. The
function below takes in an $x, y$ position, a reference distance, a reference
sound pressure level, and a list of node events. The returned value is the
probability of the sound being at point $x, y$.  \begin{align*} \mathcal{P}(x,
y, r, spl, events) &= \frac{\mathlarger{\sum}_{e \in
events}{\Normal{\Function{D}(x, y, e)}{\Function{D}_{s}(r, spl,
e.\Var{spl})}{\Function{GetSD}(e, events)}}}{|events|} \end{align*}

Now, we have a function $P$ that can determine the probability of a sound being
in an $(x, y)$ location. Therefore, given a list of node events for a gvien
sound, and the sound's associated metadata, we can determine the position of
the sound using gradient decent assuming that there is only one instance of the
sound for all of the node events. The equation for the location of the sound is
shown below. \begin{align*} \mathcal{L}(r, spl, events) &= \argmin{x, y}
-\mathcal{P}(x, y, r, spl, events) \end{align*}

The function, $\mathcal{L}$, provides us with a sound location given that there
is a single source within the WASN for the given sound.

\subsubsection{Multiple Source}

In order to do multiple source localization, we have to bias the way we perform
gradient decent on the probability density function, $\mathcal{P}$. It is
needed to start the decent at a certain point. Below is an ammended function,
$\mathcal{L}_m$ which also takes in an initial guess for the gradient decent.
\begin{align*} \mathcal{L}_m(r, spl, events, x_g, y_g) &=
\grad{x, y}(\mathcal{-P}, x_g, y_g, r, spl, events) \end{align*}

The variables, $r$, $spl$, $events$ are the same as in $\mathcal{L}$, however,
$x_g$ and $y_g$ are the initial guesses for the gradient decent. The function
returns the $x$ and $y$ values that minimizes the equation $-\mathcal{P}$ given
constant arguments $r$, $spl$, and $events$.

Now that we are able to bias the gradient decent, it is possible for us to find
mutliple local maximas in the probability density function, $\mathcal{P}$. This
is done by iterating through the list of events and performing the gradient
decent function with an initial guess of the position of the event. Once we
have a list of $x, y$ points, we are able to apply an affinity propogation
algorithm to determine the point clusters and therefore distinguish between
multiple sources of the same sound in our WASN. This is shown in
\textbf{Algorithm} \ref{algo:MS}.

\begin{algorithm}[ht] \caption{Multiple Source Sound Localization}
\label{algo:MS} \begin{algorithmic}[1] \setcounter{ALC@line}{0}

\vspace*{1mm}

\STATE $\Function{GetLocations}(\Var{r}, \Var{spl}, \Var{eventList}) \rightarrow$
\STATE $\Var{locations} \leftarrow \Function{InitLocationList}()$
\FORALL {$e \in \Var{eventList}$}
\STATE $\Var{locations}.\Var{add}(\mathcal{L}_m(\Var{r}, \Var{spl}, \Var{eventList}, e.\Var{x}, e.\Var{y}))$
\ENDFOR
\STATE $\Var{centers} \leftarrow \Function{AffinityPropagation}(\Var{locations})$
\COMMENT {\Var{centers} is a list of $(x, y, c)$ points where $c$ is the confidence}
\RETURN $\Var{centers}$

\end{algorithmic}
\end{algorithm}

Now given reference data and a list of node events, we it is possible to
determine the position of multiple instances of the same sound.

\end{document}
