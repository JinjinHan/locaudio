
\title{Probabilistic Audio Localization in Wireless Acoustic Sensor Networks}

\date{\today}

\author{Alexander Wallar}

\documentclass[12pt]{article}

\usepackage[pdftex]{graphicx}

\usepackage{amsmath}

\usepackage{float}

\floatstyle{ruled} \newfloat{program}{thp}{lop} \floatname{program}{Structure}

\newcommand{\Normal}[3]{\mathcal{N}(#1, #2, #3)}

\newcommand{\Acronym}[1]{\ensuremath{{\small{\texttt{#1}}}}}
\newcommand{\Name}{\Acronym{Camgaze.js}} \newcommand{\False}{\Constant{false}}
\newcommand{\True}{\Constant{true}}
\newcommand{\Symbol}[1]{\ensuremath{\mathcal{#1}}}
\newcommand{\Function}[1]{\ensuremath{{\small \textsc{#1}}}}
\newcommand{\Constant}[1]{\ensuremath{\small{\texttt{#1}}}}
\newcommand{\Var}[1]{\ensuremath{{\small{\textsl{#1}}}}}
\newcommand{\argmin}[1]{\underset{#1}{\operatorname{arg}\,\operatorname{min}}\;}

\begin{document}

\maketitle

\section{Implementation}

\subsection{Acquisition}

To determine the position of a sound within a wireless acoustic sensor network
(WASN), we must discuss the structure and type of data that is being sent from
the acoustic nodes. We define a \emph{node event}, which is the structure being
sent from the acoustic nodes to central server. Below is an outline of a node
event.


\begin{program}
\caption{Node Event Structure}
\begin{verbatim}
{
    x: <Float: X position>,
    y: <Float: Y position>,
    spl: <Float: Sound pressure level>,
    timestamp: <Float: Unix time in seconds>,
    confidence: <Float: Confidence of recognition>
}
\end{verbatim}
\end{program}

The node event structure provides all of the acquisition data needed by the
central server in order to perform audio localization. This structure is sent
in JSON to the central server using a \verb|POST| request upon the positive
recognition of a sound in the server's database. The recognition is done on the
wireless acoustic sensor node.

\subsection{Localization}

In order to conduct localization using the node events, we have created a
probability distribution that given an $x$ and $y$ position along with a list
of node events and reference data regarding the detected sound (such as sound
pressure level and distance at time of recording), we can deduce the
probability of the sound being at position $(x, y)$. The probability function
is fairly complex and is broken down into subfunctions below.

Firstly, at the heart of the probability function is a normal distribution
denoted as $\Normal{X}{\mu}{\sigma}$. To determine the mean for the normal
distribution, we used the predicted distance from node that the sound will be
given the reference sound pressure level and distance at which the reference
data was taken as well as the acquired sound pressure level. With these three
pieces of data, we are able to predict the distance away from the microphone
the sound was. This gives us the mean value for the normal distribution. The
function is shown below.  \begin{align*} \Function{dist}_{sound}(r, spl, spl')
&= r \cdot 10 ^ {\frac{(spl - spl')}{20}} \end{align*}



\end{document}
